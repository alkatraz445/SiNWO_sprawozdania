\documentclass[a4paper,12pt]{article}
\usepackage{geometry}
 \geometry{
 a4paper,
 total={170mm,257mm},
 left=20mm,
 top=20mm,
 }
\usepackage{adjustbox}
\usepackage[polish]{babel}
\usepackage{polski}
\usepackage{boldline}
\usepackage[T1]{fontenc} 
\usepackage{listings}
\usepackage{color}
\usepackage{biblatex}
\usepackage{csquotes}
% \usepackage{indentfirst}
\usepackage{hyperref}
\usepackage{wrapfig}
\usepackage{amsmath}
\usepackage{xcolor}
\usepackage{tcolorbox}

\newcommand*{\TakeFourierOrnament}[1]{{%
\fontencoding{U}\fontfamily{futs}\selectfont\char#1}}
\newcommand*{\danger}{\TakeFourierOrnament{`1}}

\hypersetup{colorlinks,
    citecolor=black,
    filecolor=black,
    linkcolor=black,
    urlcolor=black,
    pdftitle={WSL2 - Windows Subsystem for Linux 2},
}

\title{WSL2 \\ \large{Windows Subsystem for Linux 2}}
\date{}
\author{Jakub Kraus}
\begin{document}
\maketitle
\section{Cel i zakres ćwiczenia}
Celem ćwiczenia jest zapoznanie się z narzędziem WSL2, które pozwala na uruchamianie systemu Linux wewnątrz systemu Windows.
\section{Co to jest WSL2?}
WSL w wersji 2 jest w skrócie narzędziem działającym jako frontend dla hyperV, który zajmuje się menadżowaniem maszyn wirtualnych na systemach Microsoft Windows. W przeciwieństwie do WSL1, które było tylko warstwą kompatybilności (tłumaczyło polecenia linuxowe na polecenia windowsowe), WSL2 jest w pełni zwirtualizowaną maszyną
co pozwala na własny pełnoprawny kernel linuxowy (aktualnie w wersji 5.15LTS).
\section{Charakterystyka narzędzia WSL2}
\subsection{Zalety}
\begin{itemize}
    \item W pełni zwirtualizowana maszyna
\end{itemize}
\subsection{Wady}
\section{Proces wdrażania WSL2 na sytemie Microsoft Windows}
\subsection{Wymagania systemowe}
Aby móc korzystać z WSL2 należy spełnić kilka warunków:
\begin{itemize}
    \item System Windows 10 w wersji $\geq$ 2004 lub Windows 11
    \item Włączona wirtualizacja w UEFI
    \item Włączona opcja WSL2
    \item Włączona opcja Hyper-V
\end{itemize}
\subsubsection{Włączenie wirtualizacji w UEFI}
Włączenie wirtualizacji w UEFI jest zależne od producenta płyty głównej. W większości przypadków jest to opcja włączana w UEFI w zakładce \textbf{Security}.

\begin{tcolorbox}
    \begin{minipage}{1\textwidth}
        \large{\danger}
        Z uwagi na różnorodność producentów płyt głównych nie jestem w stanie podać jednoznacznej instrukcji jak włączyć wirtualizację w UEFI. Najlepiej jest skorzystać z instrukcji producenta swojej płyty głównej. Przykładowo strona lenovo: \url{https://support.lenovo.com/pl/pl/solutions/ht500006}
    \end{minipage}
\end{tcolorbox}

Aby dotrzeć do UEFI należy uruchomić komputer ponownie i wcisnąć klawisz odpowiedzialny za wejście do UEFI (zazwyczaj jest to klawisz F2, F10, F12 lub DEL).

\subsection{Instalacja narzędzia WSL2}
Jeśli wymagania są spełnione jedyne co jest wymagane do instalacji WSL2 to uruchomienie PowerShell jako administrator i wykonanie poniższej komendy:
\begin{lstlisting}
wsl - -install
\end{lstlisting}
To polecenie powinno włączyć wszystkie wymagane opcje i zainstalować Ubuntu w wersji LTS (na ten moment 22.04) na naszym komputerze.
\section{Instalacja systemu Ubuntu}
\section{Uruchomienie systemu Ubuntu}
\section{Wykonanie poleceń w terminalu systemu Ubuntu}
\section{Program ćwiczenia}
\tableofcontents
\end{document}